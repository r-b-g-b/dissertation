\begin{abstract}
An understanding of plasticity, the ability of the brain to reorganize based on experience, is fundamental to understanding how the brain functions. Previous research has uncovered the molecular and cellular mechanisms by which neural activity leads to specific neural circuit modifications, but in order to appreciate their role in learning and memory, studies of large-scale neural systems are required. In this dissertation, I describe several studies that take advantage of mouse lines possessing mutations in genes known to be important for distinct types of plasticity.
A mouse line lacking the \textit{Fmr1} gene replicates many of the disease phenotypes of Fragile X Syndrome (FXS), the most common single-gene determinant of mental disability in humans. \textit{Fmr1} knock-out (KO) mice do not undergo the critical period for auditory map plasticity, by which auditory cortex adapts to the sound milieu experienced immediately after hearing onset. Pharmacological blockade of group I metabolic glutamate receptors (mGluR) with 2-Methyl-6-(phenylethynyl)pyridine (MPEP) rescues this critical period, in agreement with the mGluR Theory of FXS.
I also present work in a mouse that lacks tumor necrosis factor-\textalpha{} (TNF-\textalpha{}), which is critical for the expression of homeostatic regulation of synaptic strength. Auditory cortical development is disrupted in the TNF-\textalpha{} KO mouse, however auditory map expansion occurs normally. TNF-\textalpha{} is also required for \textit{in vivo} homeostatic regulation using a multi-frequency tone exposure paradigm. Early-life acoustic stimulation with multiple frequencies leads to a homeostatic decrease of spontaneous activity and narrowing of receptive field frequency bandwidths in wild-type animals, contrasted with KO animals, which experience an anti-homeostatic increase in spontaneous activity and broadening of receptive field bandwidths.
Finally, I describe a mouse model study of the role of homeostatic plasticity in tinnitus, a phantom sound percept that commonly accompanies partial hearing loss. The development of salicylate-induced and unilateral hearing loss-induced tinnitus-related behavior requires the presence of TNF-\textalpha{}. Direct application of recombinant TNF-\textalpha{} is sufficient to cause tinnitus-related behavior. In addition, reallocation of the deprived auditory cortex to ipsilateral ear inputs requires TNF-\textalpha{} expression. Single-gene knockout animals are valuable tools to extend our understanding of plasticity mechanisms from the molecular and cellular scale to the level of circuits and the intact brain.
\end{abstract}