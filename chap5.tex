\chapter{Discussion}
In the present work, I examined the critical period for plasticity in two genetic knockout models, \textit{Fmr1} KO and TNF-\textalpha{} KO. I sought to understand the link between genes, specific plasticity mechanisms, plasticity of larger-scale networks, and perception.

% Intro
In Chapters 2 and 3, I described auditory map development in mice lacking \textit{Fmr1} and TNF-\textalpha{}, respectively. Here, I will attempt to put these findings into the context of our current understanding of cortical critical periods with specific attention to how the results fit into the growing body of evidence showing the importance of inhibition in critical period expression. In Chapter 4, I turn to the auditory disorder tinnitus, which I hypothesize results from maladaptive homeostatic plasticity.

% In the auditory cortex, the earliest critical period is that for the allocation of cortical area to different sound frequencies. This has been best studied in rats, where it was found to take place between post-natal day 11 (P11) and P14 (\cite{DeVillers-Sidani2008}). During this period, simply rearing in the presence of a pure tone can dramatically and permanently increase the cortical area tuned to the exposed tone and can affect perception long into adulthood (\cite{Han2007}). The studies in the current work largely pertain to this critical period, although it should be noted that subsequently, auditory cortex undergoes critical periods for more complex sound features, such as sweep direction and binaural cues (\cite{Insanally2009, Popescu2010a}).

\section{Inhibition and Critical Periods}
% inhibition is important, Fagiolini 2000 study
The precise mechanism underlying the expression of critical periods has long remained a mystery, although recent work has revealed that the maturation of inhibitory circuits plays a crucial role. Some evidence for the importance of inhibition comes from experiments that have prematurely opened or that delayed the critical period for ocular dominance plasticity. In wild-type mice, depriving vision from one eye early in life causes a drastic remapping of visual cortex to respond to the open eye, whereas similar monocular deprivation later in life has no lasting effect. In mice lacking \textit{Gad65}, one of two enzymes responsible for synthesizing the inhibitory neurotransmitter GABA, monocular deprivation is unable to cause ocular dominance shifts at any age, indicating that the gene is necessary for critical period expression (\cite{Fagiolini2000}). Importantly, supplementing inhibition with benzodiazepine GABA$_\mathrm{A}$ agonists immediately initiates the critical period in \textit{Gad65} KO mice, even much later in life. Furthermore, the same benzodiazepam treatment leads to precocious critical period in wild-type mice (\cite{Fagiolini2000}). Another study using a mouse line with elevated levels of brain-derived neurotrophic factor (BDNF), which accelerates maturation of inhibitory neurons, found that the critical period for ocular dominance is also accelerated (\cite{Hanover1999, Huang1999}). These studies suggest that critical period learning is permitted when inhibitory tone passes a particular threshold level.

% Fagiolini 2004, parvalbumin-positive cells are important
Inhibitory neurons are not a homogeneous class of cells but rather come in a variety of different types, which presumably perform different roles in the brain. Across multiple studies, the fast-spiking parvalbumin-positive interneurons appear to play a crucial role in critical period expression. These neurons synapse on the soma of pyramidal cells where they provide rapid feed-forward and feed-back inhibition (\cite{Markram2004}). Building off of the previously mentioned finding that pharmacological activation of GABA$_\mathrm{A}$ receptors leads to a precocious critical period, experimenters took advantage of genetic ``knock-in" mouse lines in which GABA$_\mathrm{A}$ receptors containing certain subunits are rendered insensitive to diazepam to selectively activate putative subpopulations of inhibitory neurons. Different classes of inhibitory neurons make contact at synapses enriched for GABA$_\mathrm{A}$ receptors containing particular subunits, e.g. GABA$_\mathrm{A}$ receptors containing \textalpha{}1 subunits are preferentially located on the soma post-synaptic to parvalbumin-positive inhibitory neurons (\cite{Fagiolini2004}). This additional information enabled the experiment to address whether critical period expression requires activity in a particular class of inhibitory neurons versus inhibition in general.  Premature initiation of the critical period is only possible when \textalpha{}1 subunit-containing GABA$_\mathrm{A}$ receptors are activated, suggesting that parvalbumin-positive networks play an important role in the critical period.

% studies of circuits in auditory cortex
Inhibitory neurons appear to be important for critical periods in the auditory cortex as well. In fact a more detailed understanding of the role of inhibition comes from experiments that measure inhibitory and excitatory currents in the auditory cortex before, during, and after the critical period. In the adult auditory cortex, the frequency-intensity tuning of excitation and inhibition are precisely matched, i.e. transient auditory stimuli elicit a consistent pattern of excitation followed by rapid co-tuned inhibition (\cite{Wehr2003, Wehr2005}). Before the critical period for map plasticity, inhibitory circuits are not co-tuned and are not activated with such temporal precision, but after brief exposure to patterned sound stimuli, excitation and inhibition become co-tuned and resistant to further changes, bearing greater resemblance to the adult circuit (\cite{Dorrn2010}).

% Exposure to broadband noise prevents the normal maturation of parvalbumin-positive interneurons.
Even more convincing evidence comes from the discovery that noise exposure causes cortical networks to undergo a reversal of inhibitory circuit maturation marked by reduced GABA$_\mathrm{A}$-\textalpha{}1 and \textbeta{}2/3 subunits and reduced BDNF levels, which as already mentioned are important in the development of inhibitory circuits. Functional indicators of maturational state are also altered, including the re\"emergence of broad excitatory receptive fields, poor following of temporally modulated sounds, and decreased neural synchrony. Importantly, following cessation of noise exposure, cortex in these animals regains the ability to undergo pure-tone evoked map expansion (\cite{Zhou2011}).

\section{\textit{Fmr1} and Critical Periods}
Given the wealth of evidence that development of inhibitory circuits is important for expression of a critical period, it is important to interpret the results of the current work in terms of effects on inhibition. In Chapter 2, I presented results that indicate the critical period for auditory map plasticity is impaired in mice lacking \textit{Fmr1}. Other studies in this model organism confirm that they indeed show signs of altered inhibition, or more specifically, disrupted excitatory-inhibitory balance (\cite{Gibson2008}). Perhaps the most salient indication of malfunctioning inhibition in Fragile X syndrome is the increased susceptibility to audiogenic seizures, both in human patients and in animal models (\cite{Hagerman, Chen2001}). This observation led to many studies finding weakened inhibition in FXS, although given the more recent observation of links between inhibition and critical periods, the results can be interpreted more broadly in terms of early learning and circuit formation (\cite{ElIdrissi2005}). Overall protein levels of GABA$_\mathrm{A}$ receptor \textalpha{}1, \textbeta{}2, and \textdelta{} subunits are reduced in \textit{Fmr1} KO forebrain, as well as important enzymes related to GABA metabolism, GABA transaminase and succinic semialdehyde dehydrogenase (\cite{Adusei2010}). As mentioned previously, the \textalpha{}1 subunit is necessary for expression of the critical period for ocular dominance, so if we generalize the finding to auditory cortex, it could contribute to our observed critical period impairment. Consistent with ocular dominance plasticity studies implicating parvalbumin-positive associated \textalpha{}1-containing GABA$_\mathrm{A}$ receptors, \textit{Fmr1} KO mice display a 20\% reduction in the density of parvalbumin-positive neurons in neocortex, while they show no significant decrease in other inhibitory subtypes, including calbindin- and calretinin-positive neurons (\cite{Selby2007}).

To date, the most promising explanation put forth for the molecular basis of FXS is the ``mGluR Theory of Fragile X Syndrome" (\cite{Bear2004}). FMRP, the protein product of the \textit{Fmr1} gene, is an important regulator of protein expression through its interactions with a wide variety of mRNA transcripts (\cite{Brown2001}). One of the functions of FMRP is to inhibit the mGluR signalling pathway, which when activated can lead to a complex set of downstream events, including hippocampal LTD, corticostriatal LTP, fear memory formation in the amygdala (\cite{Oliet1997, Gubellini2003, Rodrigues2002}). The mGluR Theory poses that the disease phenotypes of FXS result from failure of negative feedback onto the mGluR pathway due to the absence of FMRP. In Chapter 2, we show that map plasticity is impaired in mice lacking \textit{Fmr1}, but that MPEP, an mGluR antagonist is sufficient to restore map plasticity. Due to the multitudinous outcomes of mGluR activation, the precise mechanism by which MPEP rescues map plasticity is not clear from our study. Inhibition could still play a role, as mGluR antagonists have been reported to alleviate prolonged hippocampal epileptiform activity in a mouse model of FXS, suggesting that mGluR manipulation can correct for disrupted excitatory-inhibitory balance (\cite{Chuang2005}).

\section{TNF-\textalpha{} and Critical Periods}

I took advantage of the TNF-\textalpha{} KO mouse line, in which homeostatic plasticity is disrupted, to study the role of homeostatic plasticity in two markedly different phenomena: the critical period for map plasticity and tinnitus.
% As mentioned previously, the neonatal auditory cortex is distinct from the adult brain in many ways, including having weaker inhibition and ungated LTP.
The benefit of studying the same process at different stages of brain development is that we can gain an understanding of how the role and operation of that process changes in response to different exogenous conditions. In my research, I examined homeostatic plasticity in the face of multiple stimulation contexts and at two stages of brain development. The conditions can be summarized as: 1. early-life exposure to normal ambient auditory stimuli (Chapter 3), 2. early-life exposure to single-tone pip trains (Chapter 3), 3. early-life exposure to multi-tone pip trains (Chapter 3), and 4. adult hearing loss (Chapter 4). As mentioned before, there are multiple forms of homeostatic plasticity, including forms that do not appear to involve TNF-\textalpha{} (\cite{Stellwagen2006}). From here, unless otherwise specified, I will reserve ``homeostatic plasticity" to refer to those homeostatic mechanisms depending on TNF-\textalpha{}.

In Chapter 3, I presented data that show how the absence of TNF-\textalpha{} impairs auditory cortical development in a normal sound environment while pure-tone evoked auditory map expansion is intact. There are many steps that give rise to the tonotopic map. In mice, precise tonotopy develops before hearing onset, and depends on patterned spontaneous calcium spikes generated in the spiral ganglia and driven by cholinergic cells in the medial olive (\cite{Elgoyhen1994, Cao2008, Clause2014}). Later, experience-driven refinements of the map occur (\cite{DeVillers-Sidani2008, Han2007}). It is not yet clear which of these stages relies on TNF-\textalpha{} since the knockout animals lack the protein throughout life. Exposure to single-tone pip trains begining at hearing onset normalized the frequency range to a degree. If we interpret the mechanism for this to be that tone exposure boosted cortical activity to within the homeostatic set range, eliminating the need for homeostatic plasticity (as explained later), it would indicate that  homeostatic plasticity during in the pre-hearing period is not strictly necessary for development of normal maps. This is fairly circuitous evidence, but it at least provides a clue to the true answer. Early auditory cortex mapping and temporal control of gene expression, for example using a Tet system (\cite{Gossen1995}), could reveal which time points require TNF-\textalpha{} expression. A future experiment could turn on TNF-\textalpha{} during development and off during the critical period. If such a manipulation led to normal maps, it would suggest that homeostatic plasticity is only needed during pre-hearing development.

Map development in standard mouse husbandry conditions was altered in TNF-\textalpha{} KO mice, which we found to have incomplete representations of the normal mouse hearing range. One could imagine that the ``standard" husbandry environment is an impoverished or unnatural sound milieu. Still, the important observation in our study is that in the same environment (regardless of whether or not it can be characterized as impoverished), WT animals develop maps that represent the full range of frequencies, with ordered tonotopy, in clear contrast to KO mice. Even if we accept that there is less sound energy in the husbandry environment, it only highlights the importance of homeostatic plasticity in normal map development. In the wild, it is reasonable to assume that average sound energy experienced by different animals and litters varies considerably, making mechanisms like TNF-\textalpha{}-dependent homeostatic plasticity critically important for their ability to regulate activity to a level permissive of normal map development.

In contrast, early-life exposure to single-tone pip trains did not depend on homeostatic plasticity, as indicated by the fact that exposure led to expansion of the area representing the exposure frequency in KO mice. In addition, even though the exposure consisted of a single frequency (25 kHz), it expanded the range of frequencies represented in individual KO maps compared to those in unexposed KO mice (Chapter 3, Figure 3). There are a few possibilities for how narrow-band stimulation might have affected this global map characteristic. Auditory cortical neurons respond to a broad range of frequencies. In our study, the average tuning bandwidth was $2.7\pm0.2$ octaves, and in the developing brain bandwidths are generally even larger (\cite{Zhang2001}). Using the conservative average value from our study, a neuron tuned to 9.8 kHz could still have been significantly activated by the exposure stimulus during development. In addition to this fact, map expansion was concurrently taking place, meaning that an even greater area of auditory cortex was activated by the exposure stimulus. This suggests the possibility that abnormally low activity levels in the KO brain prevented normal map development. Indeed, spontaneous and evoked firing rates were lower in KO animals than WT animals (Chapter 3, Figure 4F \& G). In this scenario, the pure-tone stimulus drove activity across a wide swath of cortex, and the elevated activity was sufficient to bring about development of the normal tonotopy. In this stimulus regime, homeostatic plasticity was unnecessary since stimulus conceivably boosted activity across auditory cortex.

Combining the \textit{Fmr1} and TNF-\textalpha{} studies gives us a clearer picture of how Hebbian and homeostatic mechanisms might interact to allow the critical period for auditory plasticity unfold. In an auditory environment enriched with a pure-tone, neurons with some preference for that tone will over the course of exposure be repeatedly driven by that stimulus. This could engage Hebbian mechanisms to strengthen synapses activated by the exposed frequency and re-tune the neurons, leading to a macroscopic map expansion. This hypothesis is consistent with the absence of map expansion in \textit{Fmr1} KO mice, which lack the ability to undergo cortical LTP. It also is in line with the finding that in TNF-\textalpha{} KO mice, which are deficient in homeostatic plasticity but possess the ability to undergo cortical LTP, map expansion progresses normally. As discussed previously, LTP is likely to be a key plasticity mechanism in the critical period for auditory map plasticity as its ability to be induced at thalamocortical synapses is correlated with the timing of the critical period (\cite{Chun2013}). The picture is complicated by the fact that MPEP is able to restore auditory map plasticity, but does not restore cortical LTP (\cite{Wilson2007}). One possibility is that the conflicting results reflect a difference in \textit{in vivo} and \textit{in vitro} experimental methods---in our study, MPEP was administered for many days over the course of sound exposure, whereas in a slice preparation the drug is applied for a relatively short time.

A clear \textit{in vivo} realization of theoretical principles of plasticity comes from the broadband stimuluation regime of the enriched environment (EE), in which neurons have to contend with competition from multiple strong inputs. A simple plasticity model governed by LTP would predict that strong repeated stimulation would trigger LTP and strengthen inputs from a wide range of frequencies, leading to increased receptive field bandwidths. In fact, only in KO animals do bandwidths increase, whereas in WT animals bandwidths actually decrease. The KO condition stands as a clear example of the problem of LTP in the absence of homeostatic plasticity in which positive-feedback leads to synapse strengthening towards saturation. In WT mice, the homeostatic mechanism of synaptic scaling decreases the engagement of LTP for more weakly associated inputs while still allowing LTP for inputs that strongly drive the neuron. Together, LTP and homeostatic plasticity produce a system in which competition between inputs can lead to refinement of selectivity, which is important for perceptual discrimination of stimuli (\cite{Han2007}).

\section{Tinnitus and plasticity}

Critical period learning is but one of the many phenomena brought about by neural plasticity. The brain is plastic throughout life, and as I discussed previously, adult plasticity shares many of the specific cellular and molecular mechanisms that underlie critical period learning. We generally think of plasticity as a desirable trait for a neural network to possess---the ability to incorporate past experiences into future decisions can confer significant survival and reproduction advantages on an organism. However, the immense power plasticity wields over brain function can under certain circumstances lead to undesired consequences. ``Maladaptive plasticity" is simply any plasticity that results in impaired brain function, and is thought to be involved in phantom perception following limb amputation, motor remapping after a stroke, and, as I demonstrate in Chapter 4, tinnitus following hearing loss (\cite{Flor2006, Takeuchi2012}).

Hearing loss brings about drastic changes to cortical input, which leads to major reorganization of cortical circuits. High-frequency hearing loss leads to map expansion of the spared, lower frequency representation and decreased inhibition of high frequency zones (\cite{Yang2013}). This homeostatic manipulation of synapse strength might depend on TNF-\textalpha{}, since blocking TNF-\textalpha{} signalling using a soluble TNF receptor prevents downregulation of mIPSCs, a measurement indicative of the strength of inhibitory synapses (\cite{Stellwagen2006}). Further studies will be needed to directly compare changes in inhibitory synaptic strength following noise-induced hearing lesion in TNF-\textalpha{} WT and KO animals.

In Chapter 4, we also showed that unilateral hearing loss leads to reallocation of the contralateral cortex to respond to ipsilateral inputs in wild-type animals and that the reallocation is impaired in TNF-\textalpha{} KO animals. To reiterate, WT mice experience tinnitus-related behavior following unilateral hearing lesion, while KO animals do not. It is unclear if a causal relationship remapping relates and perception of tinnitus exists, despite the correlation between the two. It is interesting how strikingly similar the result is to that observed using the monocular deprivation paradigm. In both cases, unilateral sensory deprivation leads to deprived neurons retuning to ipsilateral inputs, and furthermore, absence of TNF-\textalpha{} prevents the remapping from occurring (\cite{Mrsic-Flogel2007, Kaneko2008}). We can attempt to draw some insight into our result from the additional research performed in the visual system. In the visual cortex, the first stage involves a decrease in responses to the deprived side and is independent of TNF-\textalpha{}. The second stage, in which responses to the spared side increase, requires TNF-\textalpha{}. Importantly, in the second stage, synapse strengthening occurs even for deprived eye inputs, which remain deprived. This suggests that synapse strengthening is not input-specific, and strongly indicates that it proceeds through the homeostatic synaptic scaling shown previously to depend on TNF-\textalpha{}. A similar study is needed in auditory cortex to find out if unilateral hearing lesion occurs through the same plasticity mechanisms.

\section{Interactions between plasticity mechanisms}
For practical reasons, experimental manipulations often focus on increasing or decreasing levels of one protein. This is the paradigm used in all of the individual experiments used in this study. If one draws conclusions from the converging results of many single manipulations, it is possible to come up with plausible models for how the brain actually operates. Still, it is important to keep in mind that, however pragmatic it might be to partition biological systems into individual components, biological systems are highly interconnected. To that point, there are some recent studies that show interactions between homeostatic and Hebbian forms of plasticity. Researchers looking at Schaffer collateral-CA1 synapses in the hippocampus showed that induction of homeostatic plasticity enhances the strength of subsequently induced LTP (\cite{Arendt2013}). In addition to eliciting robust homeostatic plasticity via insertion of new AMPARs in dendritic spines, action potential blockade using tetrodotoxin (TTX) also promotes the formation of new NMDA-only ``silent" synapses, so-called because they do not pass current following simple stimulation. However, ``silent" synapses are converted to active synapses after LTP induction, so by increasing the number of silent synapses, homeostatic plasticity primes the cell to undergo stronger LTP. There is also some evidence that a local homeostatic plasticity mechanism is activated following LTP induction (\cite{Rabinowitch2008}). This mechanism achieves homeostasis, not by globally altering strengths of all synapses but by weakening synapses close to a synapse that underwent LTP.

In my interpretation of my critical period studies, I drew heavily on the current theory that inhibitory tone is a key determinant of the ability of cortex to undergo critical period learning. Still, despite a large body of evidence showing that inhibition is important, there is still no clear mechanism for how immature inhibitory networks allow critical period expression. One possibility is that weakened inhibition simply permits elevated activity levels that are favorable to Hebbian activity-dependent plasticity. One study suggests that a mutant line with enhanced inhibition also loses the ability to undergo LTP in the dentate gyrus. Tamping down inhibition using bicuculine, a GABA$_\mathrm{A}$ receptor antagonist restored the ability to undergo LTP, and in addition, LTP could be blocked in wild-type mice by administration of diazepam, a GABA$_\mathrm{A}$ receptor agonist (\cite{Levkovitz1999}). In auditory cortex, there also seems to be a link between disinhibition and LTP. While LTP of thalamocortical synapses is readily elicited in neonates during the critical period, the same protocol fails to elicit LTP in adults. However, removing inhibitory currents using intracellular injection of picrotoxin in cortical pyramidal neurons or induction of LTD at inhibitory synapses rendered the thalamocortical synapses capable of undergoing LTP. Group I metabolic glutamate receptors appeared to underlie this plasticity since MPEP prevented the induction of LTP even with reduced inhibition (\cite{Chun2013}). Studies like these show the importance of considering how the many uniquely identified conceptual elements of the brain are interrelated, and future studies stand to make significant progress by considering how different processes interact.