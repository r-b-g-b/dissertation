\chapter{Discussion}

\section{Genetic Knockout Models and Critical Period Plasticity}

In the present work, we examined the critical period for plasticity in two genetic knockout models, \textit{Fmr1} KO and TNF-\textalpha{} KO. We sought to understand the link between genes, specific plasticity mechanisms, plasticity of larger-scale networks, and perception.

% Intro
In Chapters 2 and 3, I described auditory map development in mice lacking \textit{Fmr1} and TNF-\textalpha{}, respectively. Here, I will attempt to put these findings into the context of our current understanding of cortical critical periods. In the auditory cortex, the earliest critical period is that for the allocation of cortical area to different sound frequencies. This has been best studied in rats, where it was found to take place between post-natal day 11 (P11) and P14 (\cite{DeVillers-Sidani2008}). During this period, simply rearing in the presence of a pure tone can dramatically and permanently increase the cortical area tuned to the exposed tone and can affect perception long into adulthood (\cite{Han2007}). The studies in the current work pertain to this critical period, although it should be noted that subsequently, auditory cortex undergoes critical periods for more complex sound features, such as sweep direction and binaural cues (\cite{Insanally2009, Popescu2010a}).

\subsection{Inhibition and Critical Periods}
% inhibition is important, Fagiolini 2000 study
The precise mechanism underlying the expression of critical periods has long remained a mystery, although recent work has revealed that the maturation of inhibitory circuits to play a crucial role. Some evidence for the importance of inhibition comes from experiments that have prematurely opened or that delayed the critical period for ocular dominance plasticity. In wild-type mice, depriving vision from one eye early in life causes a drastic remapping of visual cortex to respond to the open eye, whereas similar monocular deprivation later in life has no lasting effect. In mice lacking \textit{Gad65}, one of two enzymes responsible for synthesizing the inhibitory neurotransmitter GABA, monocular deprivation is unable to cause ocular dominance shifts at any age, indicating that the gene is necessary for critical period expression (\cite{Fagiolini2000}). Importantly, supplementing inhibition with benzodiazepine GABA$_\mathrm{A}$ agonists immediately initiates the critical period in \textit{Gad65} KO mice, even much later in life. Furthermore, the same benzodiazepam treatment leads to precocious critical period in wild-type mice (\cite{Fagiolini2000}). Another study using a mouse line with elevated levels of brain-derived neurotrophic factor (BDNF), which accelerates maturation of inhibitory neurons, found that the critical period for ocular dominance is also accelerated (\cite{Hanover1999, Huang1999}). These studies suggest that critical period learning is permitted when inhibitory tone passes a particular threshold level.

% Fagiolini 2004, parvalbumin-positive cells are important
Inhibitory neurons are not a homogeneous class of cells but rather come in a variety of different types, which presumably perform different roles in the brain. Across multiple studies, the fast-spiking parvalbumin-positive interneurons appear to play a crucial role in critical period expression. These neurons synapse on the soma of pyramidal cells where they provide rapid feed-forward and feed-back inhibition (\cite{Markram2004}). Building off of the previously mentioned finding that pharmacological activation of GABA$_\mathrm{A}$ receptors leads to a precocious critical period, experimenters took advantage of the additional fact that different inhibitory subtypes make contact at synapses enriched for GABA$_\mathrm{A}$ receptors containing particular subunits. For example, GABA$_\mathrm{A}$ receptors containing \textalpha{}1 subunits are preferentially located on the soma post-synaptic to parvalbumin-positive inhibitory neurons. Along with the development of genetic ``knock-in" mouse lines in which certain GABA$_\mathrm{A}$ receptors containing certain subunits are rendered insensitive to diazepam, experimenters were able to selectively activate putative subpopulations of inhibitory neurons to address whether critical period expression requires activity in a particular class of inhibitory neurons versus inhibition in general (\cite{Fagiolini2004}). They found that premature initiation of the critical period was only possible when \textalpha{}1 subunit-containing GABA$_\mathrm{A}$ receptors are activated, suggesting that parvalbumin-positive networks play an important role in the critical period.

% studies of circuits in auditory cortex
Inhibitory neurons appear to be important for critical periods in the auditory cortex as well, and in fact a more detailed understanding of the role of inhibition comes from experiments that measure inhibitory and excitatory currents in the auditory cortex before, during, and after the critical period. In the adult auditory cortex, the frequency-intensity tuning of excitation and inhibition are precisely matched, i.e. transient auditory stimuli elicit a consistent pattern of excitation followed by rapid co-tuned inhibition (\cite{Wehr2003, Wehr2005}). Before the critical period for map plasticity, inhibitory circuits are not co-tuned and are not activated with such temporal precision, but after brief exposure to patterned sound stimuli, excitation and inhibition become co-tuned and resistant to further changes, bearing greater resemblance to the adult circuit (\cite{Dorrn2010}).

% Exposure to broadband noise prevents the normal maturation of parvalbumin-positive interneurons.
Even more convincing evidence comes from the discovery that noise exposure causes cortical networks to revert to an infant-like state, including a reversal of inhibitory circuit maturation marked by reduced GABA$_\mathrm{A}$-\textalpha{}1 and \textbeta{}2/3 subunits and reduced BDNF levels, which as already mentioned are important in the development of inhibitory circuits. Functional indicators of maturational state were also altered, including the re\"emergence of broad excitatory receptive fields, poor following of temporally modulated sounds, and decreased neural synchrony. Importantly, following cessation of noise exposure, cortex in these animals regains the ability to undergo pure-tone evoked map expansion (\cite{Zhou2011}).

\subsection{\textit{Fmr1} and Inhibition}
Given the wealth of evidence that development of inhibitory circuits is important for expression of a critical period, it is important to interpret the results of the current work in terms of effects on inhibition. In Chapter 2, I presented results that indicate the critical period for auditory map plasticity is impaired in mice lacking \textit{Fmr1}. Other studies in this model organism confirm that they indeed show signs of altered inhibition, or more specifically, disrupted excitatory-inhibitory balance (\cite{Gibson2008}). Perhaps the most salient indication of malfunctioning inhibition in Fragile X syndrome is the increased susceptibility to audiogenic seizures, both in human patients and in animal models (\cite{Hagerman, Chen2001}). This observation led to many studies focused on weakened inhibition in FXS, although given the more recent observation of links between inhibition and critical periods, the results can be interpreted more broadly in terms of early learning and circuit formation (\cite{ElIdrissi2005}). Overall protein levels of GABA$_\mathrm{A}$ receptor \textalpha{}1, \textbeta{}2, and \textdelta{} subunits are reduced in \textit{Fmr1} KO forebrain, as well as important enzymes related to GABA metabolism, GABA transaminase and succinic semialdehyde dehydrogenase (\cite{Adusei2010}). As mentioned previously, the \textalpha{}1 subunit is necessary for expression of the critical period for ocular dominance, so if we allow the generalization of the finding to auditory cortex, it could explain our observed critical period impairment. Consistent with ocular dominance plasticity studies implicating parvalbumin-positive associated \textalpha{}1-containing GABA$_\mathrm{A}$ receptors, \textit{Fmr1} KO mice display a 20\% reduction in the density of parvalbumin-positive neurons in neocortex, while they show no significant decrease in other inhibitory subtypes, including calbindin- and calretinin-positive neurons (\cite{Selby2007}).

There is still no clear mechanism for how immature inhibitory networks allow critical period expression. One possibility is that weakened inhibition simply permits elevated activity levels that are favorable to Hebbian activity-dependent plasticity. One study suggests that a mutant line with enhanced inhibition also loses the ability to undergo LTP in the dentate gyrus. Tamping down inhibition using bicuculine, a GABA$_\mathrm{A}$ receptor antagonist restored the ability to undergo LTP, and in addition, LTP could be blocked in wild-type mice by administration of diazepam, a GABA$_\mathrm{A}$ receptor agonist (\cite{Levkovitz1999}). In auditory cortex, there also seems to be a link between disinhibition and LTP. While LTP of thalamocortical synapses is readily elicited in neonates during the critical period, the same protocol fails to elicit LTP in adults. However, removing inhibitory currents using intracellular injection of picrotoxin in cortical pyramidal neurons or induction of LTD at inhibitory synapses rendered the thalamocortical synapses capable of undergoing LTP. Group I metabolic glutamate receptors appeared to underlie this plasticity since MPEP prevented the induction of LTP even with reduced inhibition (\cite{Chun2013}).

% \subsection{TNF-\textalpha{} and Critical Periods}
In Chapter 3, I presented data that show how the absence of TNF-\textalpha{} affects development in a much different way from the absence of FMRP, namely, auditory cortical development in a normal sound environment is impaired while pure-tone evoked auditory map expansion is intact. There are many steps that give rise to the tonotopic map. In mice, precise tonotopy develops before hearing onset, and depends on patterned spontaneous calcium spikes generated in the spiral ganglia and driven by cholinergic cells in the medial olive (\cite{Elgoyhen1994, Cao2008, Clause2014}).

Combining the two studies gives us a clearer picture of how Hebbian and homeostatic mechanisms might interact to allow the critical period for auditory plasticity unfold. In an auditory environment enriched with a pure-tone, neurons with some preference for that tone will over the course of exposure be repeatedly driven by that stimulus. This could engage Hebbian mechanisms to strengthen synapses activated by the exposed frequency and re-tune the neurons, leading to a macroscopic map expansion. This hypothesis is consistent with the absence of map expansion in \textit{Fmr1} KO mice, which lack the ability to undergo cortical LTP. It also is in line with the finding that in TNF-\textalpha{} KO mice, which are deficient in homeostatic plasticity but possess the ability to undergo cortical LTP, map expansion progresses normally. As discussed previously, LTP is likely to be a key plasticity mechanism in the critical period for auditory map plasticity as its ability to be induced at thalamocortical synapses is correlated with the timing of the critical period (\cite{Chun2013}). The picture is complicated by the fact that MPEP is able to restore auditory map plasticity, but does not restore cortical LTP (\cite{Wilson2007}). One possibility is that the conflicting results reflect a difference in \textit{in vivo} and \textit{in vitro} experimental methods---in our study, MPEP was administered for many days over the course of sound exposure, whereas in a slice preparation the drug is applied for a relatively short time.

The use of genetic knockouts to study biological phenomena is 