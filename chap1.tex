\chapter{Introduction}

\section{Overview}

Plasticity, the ability to change, is a key feature of the brain's function. While much of development can be accomplished by  unfurling genetically-encoded instructions, success is more likely when the individual can bring information from past experiences to bear on its response to new stimuli, i.e. when it can learn. The past century has seen enormous progress in the study of neural plasticity, from early conceptual models to specific molecular pathways. From these studies, we have learned of two main forms of plasticity govern circuit modifications in the brain: Hebbian plasticity and homeostatic plasticity. Future advances will likely reveal how plasticity rules act and interact at the level of neural systems to generate the immense computational faculties of the brain.

\section{Early thoughts on plasticity}
The origins of a theoretical consideration of neural plasticity can be traced back to William James, who chose the term to refer to the changes that give rise to habitual behaviors \cite{James1910, Berlucchi2009}. Italian psychologists Eugenio Tanzi and Ernesto Lugaro and Spanish anatomist Santiago Ram\a`on y Cajal began to develop the idea that formation of new connections between neurons was the physiological instantiation of learning \cite{Berlucchi2009}. In his 1949 book, ``The Organization of Behavior," Donald Hebb clarified the idea of changing synpatic weights as a basis for learning. He described his model as follows \cite{Hebb1949}:

\begin{quotation}
Let us assume that the persistence or repetition of a reverberatory activity (or ``trace") tends to induce lasting cellular changes that add to its stability. [...] When an axon of cell A is near enough to excite a cell B and repeatedly or persistently takes part in firing it, some growth process or metabolic change takes place in one or both cells such that A's efficiency, as one of the cells firing B, is increased.
\end{quotation}

Hebb's rule, as this theory came to be known, charted new territory in the understanding of brain plasticity. It brought the field closer to the goal of a specific theory for how activity in a neural circuit is translated into instructions for modifying that circuit to incorporate new information.

\section{From theory to physiology}

In 1968, Timothy Bliss and Terje L\o mo discovered experimental evidence for Hebbian plasticity in rabbit dentate granule cells \cite{Bliss1973}. They observed that high-frequency electrical stimulation of the perforant path inputs onto granule cells led to those inputs being strengthened. The finding closely matched the idea of a brain that operated according to Hebb's rule---strong electrical stimulation caused the pre-synaptic neurons to fire and drive activity in the post-synaptic neurons, leading to increased synaptic weights. This implementation of Hebbian learning, which has since been replicated countless times, came to be known as ``long-term potentiation" or LTP, and serves as a foundation for how we understand neural plasticity to work.

Networks whose plasticity is governed solely by Hebbian plasticity are prone to unstable behavior. If a given synapse participates in driving the post-synaptic target to fire, its weight will be increased, causing it to be more likely to activate the post-synaptic neuron, increasing its weight, and so on. Conversely, a synapse that rarely activates the post-synaptic neuron will fall into a negative feedback loop and its weight will be reduced to zero. Not only is this an unusable learning process from a theoretical standpoint, it does not agree with experimental evidence showing the distribution of synaptic weights on a neuron to be normally distributed with a slight positive skew rather than the ``all-or-none" distibution predicted by purely Hebbian learning mechanisms \cite{Turrigiano1998}. Although Hebb described a powerful learning rule that appeared to actually operate in the brain, its instability in isolation led scientists to the hypothesis that there must be additional mechanisms at play that could ensure network stability in the brain.

That key missing component was a way to achieve activity \textit{homeostasis}, the preservation of activity (usually represented by neuronal firing rate) within a set range of values. An extraordinary number of factors determine the overall electrical activity of a neuron, and the brain seems to have evolved ways to modulate many of them to achieve homeostasis. The sodium, potassium, and calcium channels that determine a neuron's intrinsic excitability could be modulated \cite{Franklin1992}. The system could attenuate the degree of plasticity for strong synapses, preventing run-away synaptic strengthening \cite{VanRossum2000}. Another option is to increase or decrease the strengths of all synaptic inputs to the neuron to counteract deviations from the desired activity range. By multiplicitavely scaling all synapses, the cell can change its overall activity level without sacrificing the information contained in the synaptic weights. This process is known as synaptic scaling, and itself appears to have several distinct mechanisms, as will be discussed later.

With the theoretical foundations in place and the physiological mechanisms outlined, a major challenge is to understand how these plasticity rules defined at the level of pairs of neurons or small neural assemblies operate and co\"operate to affect larger scale network dynamics and, finally, behavior. The current work aims to address this question using two mouse lines deficient in important molecular components of plasticity, Fragile X mental retardation protein (FMRP) and tumor necrosis factor-\textalpha{} (TNF-\textalpha{}).

\section{\textit{Fmr1} and plasticity}

Fragile X Syndrome (FXS) is the most common cause of inherited mental retardation, resulting from the transcriptional silencing of the \textit{Fmr1} gene and the absence of its protein product, fragile X mental retardation protein (FMRP) \cite{Bailey1998, Jin2003}. Mice lacking \textit{Fmr1} display many of the same disease phenotypes as humans with Fragile X Syndrome including learning deficits, hyperactivity, auditory hypersensitivity, social impairments, and macro\"orchidism \cite{DutchBelgianFragileXConsortium1994, Bernardet2006, Moy2008}, allowing scientists a better opportunity to uncover the physiological basis of the disease.

One common neuroanatomical findings in FXS is an increased density of thin, under-developed dendritic spines \cite{Hinton1991, Comery1997, Dolen2007, Liu2011} (although some recent studies call into question whether increased spine density is a reliable pathology \cite{Cruz-Martin2010, Harlow2010a, Meredith2007}). The advent of \textit{in vivo} optical imaging of spines revealed that a hallmark of \textit{Fmr1} knock-out neurons is that their dendritic spines have increased motility and turn-over \cite{Cruz-Martin2010, Pan2010}. This observation suggests an inability for networks lacking FMRP to stabilize synaptic connections, and could imaginably explain the processing deficits seen in FXS.

FMRP performs an impressive number of roles in the cell, many of them related to the translation of dendritic mRNAs that facilitates rapid, activity-dependent synaptic alteration. One study found that FMRP associates with 432 unique mRNAs, hinting at its important regulatory role \cite{Brown2001}. Not surprisingly given its promiscuity in the cell, there are direct links between \textit{Fmr1} and several specific plasticity mechanisms. mGluR-dependent LTD is enhanced in the hippocampus, although it is normal in cortex \cite{Huber2002}. Conversely, mGluR-dependent LTP is strongly attenuated in cortex, while it is normal in the hippocampus \cite{Li2002, Zhao2005, Wilson2007}. These deficits in synaptic plasticity could underlie some of the developmental abnormalities in FXS, for example the critical period for ocular dominance, which is absent in \textit{Fmr1} KO animals \cite{Dolen2007}.

FMRP also plays a role in retanoic acid-mediated synaptic scaling resulting from action potential and NMDAR blockade \cite{Soden2010}. This form of synaptic scaling is independent of protein transcription, but instead proceeds through translation of locally-available mRNAs for GluR1-type AMPA receptors that are then inserted into the post-synaptic membrane. Interestingly, FMRP does not appear to be required for synaptic scaling when NMDARs are left unobstructed, highlighting the fact that the brain employs several redundant mechanisms to enforce homeostasis \cite{Soden2010}.

A major conceptual advance our understanding of \textit{Fmr1} and FXS came in 2004, when Bear, Huber, and Warren put forth a theory for FXS based on its interactions with mGluR. In wild-type cells, FMRP responds to mGluR5 activation by inhibiting mRNA translation in the synapse, effectively acting as negative feedback for the mGluR5 signalling pathway. Lacking FMRP, synaptic proteins are translated in excess, leading to undesirable consequences, including exaggerated long-term depression \cite{Huber2002, Bear2004}. Attenuation of mGluR signalling in mice reverses many of the abnormal phenotypes of FXS including spine morphology, impaired synaptic plasticity, protein overabundance, behavioral abnormalities, and the impaired critical period for ocular dominance \cite{DeVrij2008, Dolen2007, Su2011}.

\section{TNF-\textalpha{} and plasticity}

Tumor necrosis factor-\textalpha{} (TNF-\textalpha{}) is a protein traditionally regarded as a component of the proinflammatory response. It has recently been discovered to have a much wider range of effects, including an important role in proper neural function. In 2006, Stellwagen and Malenka discovered that synaptic scaling depends on the presence of glial TNF-\textalpha{}. Importantly, the absence of TNF-\textalpha{} had no effect on LTP and LTD, showing that the mechanisms for Hebbian and homeostatic plasticity do not significantly overlap and allowing the future studies to consider two processes with some degree of isolation \cite{Stellwagen2006}. The development of mouse line lacking TNF-\textalpha{} gave researchers the ability to study the role of homeostatic plasticity in a wide range of \textit{in vivo} preparations.

This experimental tool led to a deeper understanding of a classic model for critical period plasticity, mocular deprivation-induced shift of ocular dominance. Occluding vision in one eye early in life leads to well-characterized changes to the cells in visual cortex receiving synaptic input from both eyes. First, synaptic connections from the deprived eye become weaker, then connections from the spared eye become stronger \cite{Frenkel2004}. The two phases of ocular dominance plasticity seem to be mediated by two specific plasticity mechanisms. Loss of connections from the deprived eye results from Hebbian LTD---noisy input from the blocked eye is no longer able to reliably fire the post-synaptic neuron, thus those synapses are weakened. The increase in the synpatic strengths for spared eye inputs results from synaptic scaling---the overall lower synaptic drive leads to a reduction in firing rate for the post-synaptic neuron and the remaining synapses are up-scaled to bring activity back within the set range. In TNF-\textalpha{} mice, which lack the ability to scale up their synapses, deprived eye responses are lost, but spared eye responses never increase \cite{Kaneko2008}.

Although it is clear that TNF-\textalpha{} is necessary to express synaptic scaling, the precise relationship between TNF-\textalpha{} and homeostatic plasticity is complex and still unclear. A simple model in which decreasing activity leads to slowly accumulating release of TNF-\textalpha{} that signals insertion of additional AMPARs conflicts with the data. Instead, the effect of TNF-\textalpha{} appears to be state-dependent, leading to increases in synaptic strength when applied alone, and leading to decreases in synaptic strength when applied to prescaled synapses \cite{Steinmetz2010}. Thus, TNF-\textalpha{} does not directly signal synaptic scaling, but appears to play a permissive role, maintaining the neuron's ability to scale up synapses in the face of activity blockade \cite{Steinmetz2010}.

% Homeostatic plasticity/synaptic scaling \textit{in vivo} \cite{Hengen2013, Keck2013}
% \cite{Ranson2012}

\section{Auditory cortical map plasticity}

Throughout the animal kingdom and throughout the brain, early experiences can exert profound and permanent influence on neural circuits. While learning can take place throughout life, in many neural systems there exists a restricted time interval in which stimuli can most readily alter the structure of the underlying network. One well-studied critical period regulates allocation of neural resources in the cortical auditory map. Sound processing in the brain begins in the cochlea, a complex spiral-shaped transduction apparatus in the inner ear. The receptor cells themselves, known as hair cells for the minute mechanosensory hairs they possess, are embedded in the basilar membrane, which coils along the core of the cochlea. The mechanics of the cochlea's physical structure decomposes pressure variations in the air, so that different frequency sound inputs displace hair cells at different locations along the basilar membrane, forming a map of sound frequency. The orderly arrangement of frequency representations in space is known as tonotopy and is the most obvious feature of sound processing in the brain. Tonotopy is preserved all along the auditory pathway including primary auditory cortex (AI), where low frequency sounds activate neurons in caudal areas and high frequency sounds activate neurons in rostral areas.

The specific allocation of cortical area to different frequencies is determined during an early critical period. In rats, rearing in the presence of a 7-kHz tone between post-natal days (P) 11 and 14 leads to a significant increase in the percent of AI representing that tone. Importantly, the change persists many weeks after cessation of tone exposure, and the same exposure either before or after the P11-14 window does not lead to any significant changes to the tonotopic map \cite{DeVillers-Sidani2007}. Recent work is starting to shed light on the mechanisms underlying this critical period. Inhibitory circuits appear to play a crucial role. Around hearing onset, inhibitory currents are weak and are not co-tuned with excitation, as they are in the adult auditory cortex \cite{Dorrn2010}. In this state, pulsed tone presentation can lead to alterations in the frequency tuning of neurons, and in addition, this succeptibility is lost with the appearance of closely matched excitation and inhibition. Additional support for this theory comes from manipulations that postpone critical period closure, such as rearing in broadband noise, that also delay the normal time course of inhibitory maturation \cite{DeVillers-Sidani2008}. What it means for the network to have balanced excitation and inhibition, and why that should render it invulnerable to alteration by experience is not yet clear, but it is a promising direction in advancing our understanding critical periods.

\section{Summary of present work}

In the current work, I describe the use of several experimental techniques including single-gene knockout organisms, intracortical injection of recombinant protein, ELISA protein assay, behavioral assays, and multi-unit recording of cortical neural activity to understand the role of two key molecular players in neural plasticity, Fragile X mental retardation protein (FMRP) and tumor necrosis factor-\textalpha{} (TNF-\textalpha{}).

In Chapter 2, I present an experiment aimed at exploring the role of \textit{Fmr1} in the critical period for tonotopic map development. The study finds that in the absence of \textit{Fmr1}, the auditory map develops normally, however experience-induced map expansion does not occur. Treatment with MPEP, a drug aimed at correcting the over-active mGluR pathway, restores normal critical period plasticity.

In Chapter 3, I describe development of the auditory map in mice lacking TNF-\textalpha{}. Tonotopic map development is impaired in a sound environment that is not specifically enriched with patterened sound. Early exposure to a pulsed pure tone leads to normal map expansion, indicating that the mechanisms underlying this critical period are intact in the TNF-\textalpha{} KO mouse.

In Chapter 4, I turn to the study of tinnitus, the perception of phantom sounds. First, I show that TNF-\textalpha{} KO mice do not show tinnitus-related behaviors following noise-induced hearing loss. Then I show that intracortical injection of recombinant TNF-\textalpha{} is sufficient to cause tinnitus-related behavior in wildtype and knockout animals. I also describe the bilateral remapping of sound inputs to cortex following unilateral hearing lesion, and show that the rerouting of the intact ipsilateral input does not occur in mice lacking TNF-\textalpha{}.

\printbibliography